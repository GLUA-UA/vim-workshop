\chapter{Introdução}
\label{chap:introducao}

\section{Contextualização}
\label{sec:contextualizacao}

Nos dias de hoje os Integraded Development Environments (IDEs) são as ferramentas de desenvolvimento mais utilizadas pelos programadores e desenvolvedores de software.

Os programas mais comuns como o Visual Studio Code, JetBrains IDE's, entre outros, partilham todos dos mesmos problemas: são pesados, muitas vezes lentos, consomem muitos recursos, têm muitas ferramentas que acabam por não ser utilizadas pela maioria dos utilizadores e, finalmente, exigem demasiado uso do rato.

O Vim resolve todos estes problemas. Por padrão não tem quase nenhuma funcionalidade, mas é extremamente extensível e pode ser configurado para ser uma IDE completa, permitindo a cada utilizador a escolha das ferramente que efetivamente vai utilizar para o seu trabalho. É também bastante leve, rápido consumindo poucos recursos. O Vim por si só são apenas uns Megabytes, enquanto que editores mais utilizados são centenas de Megabytes ou até mesmo Gigabytes. Finalmente é um editor que pode ser utilizado sem qualquer uso do rato, o que melhora bastante a produtividade do utilizador.

\section{Objetivo do Guia}
\label{sec:objetivo-do-guia}

Com este guia, apresenta-se como configurar o Vim/Neovim para deixar de ser um mero editor de texto e passar a ser um ambiente utilizável para o dia a dia de um desenvolvedor de software.
