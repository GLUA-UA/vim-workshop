\chapter{Keymappings}
\label{chap:keymappings}

\section{Introdução}

O Vim é um editor de texto que tem como objetivo ser o mais eficiente possível.\ Para isso, ele permite que o utilizador crie atalhos para todo o tipo de ações.

Como o Vim é um editor de texto modal, os atalhos são diferentes em cada modo.\ Isto significa que, para uma mesma tecla, podemos definir diferentes ações para cada modo.

\section{Modos}

Os modos do Vim usados em mapeamentos são: Normal, Insert, Visual, Select, Operator-pending, Command-line e Terminal.

Para representar os modos, usamos as seguintes abreviaturas:

\begin{itemize}\label{itemize:map-modes}
    \setlength{\itemsep}{-5pt} % Adjust the value as needed
    \item \texttt{n} para o modo Normal
    \item \texttt{i} para o modo Insert
    \item \texttt{v} para o modo Visual
    \item \texttt{s} para o modo Select
    \item \texttt{x} para o modos Visual e Select
    \item \texttt{o} para o modo Operator-pending
    \item \texttt{c} ou \texttt{l} para o modo Command-line
    \item \texttt{t} para o modo Terminal
\end{itemize}

Nota: se não for especificada nenhuma abreviatura, o atalho ficará nos modos Normal, Visual, Select e Operator-pending.

\section{Map vs Noremap}
\label{sec:map-vs-noremap}

Em Vim Script existem 2 formas de criar um atalho: \texttt{map} (ou \texttt{remap}) e \texttt{noremap}.

O \texttt{map} representa mapeamento recursivo, ou seja, quando usado, o Vim irá expandir recursivamente qualquer mapeamento que seja acionado pela sequência de teclas mapeada. Isto significa que se a sequência de teclas mapeada acionar outro mapeamento, esse mapeamento será expandido quando usado \texttt{map}.\ Por outro lado, o \texttt{noremap} representa mapeamento não recursivo, ou seja, quando usado, o Vim não irá expandir recursivamente qualquer mapeamento que seja acionado pela sequência de teclas mapeada. Isto significa que se a sequência de teclas mapeada acionar outro mapeamento, esse mapeamento não será expandido quando usado \texttt{noremap}.

Geralmente, usa-se \texttt{noremap} para mapeamentos no geral e \texttt{map} para alias.

Por exemplo:

\begin{lstlisting}
:map j gg     " move o cursor para a primeira linha
:map Q j      " move o cursor para a primeira linha
:noremap W j  " move o cursor uma linha abaixo
\end{lstlisting}

Neste caso, \texttt{j} vai ser mapeado para \texttt{gg} e \texttt{Q} vai ser mapeado para \texttt{j}.\ Uma vez que \texttt{j} foi mapeado para \texttt{gg}, \texttt{Q} vai ser mapeado para \texttt{gg} uma vez que foi mapeado de forma recursiva.\ Por outro lado, \texttt{W} vai ser mapeado para \texttt{j} e não para \texttt{gg} uma vez que foi mapeado de forma não recursiva.

\section{Opções de Mapeamento}
\label{sec:map-extra-options}

As opções de mapeamento são especificadas entre \texttt{<} e \texttt{>} e podem ser usadas em qualquer ordem.

Existem as seguintes opções de mapeamento:

\begin{itemize}
    \setlength{\itemsep}{-5pt} % Adjust the value as needed
    \item \texttt{silent} suprime a exibição do comando sendo executado
    \item \texttt{expr} especifica que a função do mapeamento é uma expressão a ser avaliada
    \item \texttt{unique} especifica que o mapeamento não deve ser remapeado se já existir
    \item \texttt{script} especifica que o mapeamento deve ser local ao script onde é definido
    \item \texttt{nowait} especifica que o mapeamento não deve esperar que outros mapeamentos sejam concluídos
    \item \texttt{special} especifica que o mapeamento deve ser acionado por teclas especiais
    \item \texttt{cmd} especifica que a função do mapeamento é um comando a ser executado
    \item \texttt{args} especifica que o mapeamento deve aceitar argumentos
    \item \texttt{buffer} especifica que o mapeamento se aplica apenas ao buffer atual
\end{itemize}

\section{Sintaxe de Mapeamento}

\subsection{Vim Script}

Em Vim Script, os mapeamentos são feitos na forma:

\begin{lstlisting}
<mode>[nore]map[!] [options] <command> <function>
\end{lstlisting}

Onde:

\begin{itemize}
    \setlength{\itemsep}{-5pt} % Adjust the value as needed
    \item \texttt{<mode>} é o modo para o qual o mapeamento se aplica, devendo ser usada uma das abreviaturas listadas em \ref{itemize:map-modes}
    \item \texttt{[nore]} é opcional e indica se o mapeamento é recursivo ou não, tal como explicado em \ref{sec:map-vs-noremap}
    \item \texttt{[!]} é opcional e indica que o mapeamento deve substituir qualquer mapeamento existente para a mesma sequência de teclas
    \item \texttt{<command>} é a sequência de teclas que aciona o mapeamento
    \item \texttt{<function>} é a função a ser executado quando a sequência de teclas é acionada, que pode ser uma outra sequência de teclas, um comando ou uma função
    \item \texttt{[options]} são as opções adicionais referidas em \ref{sec:map-extra-options}, que deverão estar entre \texttt{<} e \texttt{>}
\end{itemize}

Exemplo:

\begin{lstlisting}
nnoremap <silent> <F5> :echo "Hello, World!"<CR>
\end{lstlisting}

\subsection{Lua}

Em Lua, os mapeamentos são feitos na forma:

\begin{lstlisting}
vim.keymap.set({"<mode1>", "<mode2>", ...}, "<command>", "<function>",
                {<option1> = <value1>, <option2> = <value2>, ...})
\end{lstlisting}

Em lua, os modos são passados como um set de strings, o mesmo mapeamento pode ser atribuido a vários modos numa única linha.\ Utilizando apenas 1 modo, não é necessário usar \{ \}.

A diferenciação entre \texttt{map} e \texttt{noremap} é feita através de uma opção adicional.

O equivalente ao exemplo apresentado em Vim Script:

\begin{lstlisting}
vim.keymap.set("n", "<F5>", "<cmd>echo Hello, World!<CR>", {noremap = true, silent = true})
\end{lstlisting}

