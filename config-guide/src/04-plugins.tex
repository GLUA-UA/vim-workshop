\chapter{Plugins}
\label{chap:plugins}

\section{Vim}

Para facilitar a usabilidade e atualização dos plugins, são usados plugin managers para fazer essa gestão automaticamente.

O mais comum no Vim é o Vim-Plug. Para instalar o Vim-Plug basta executar o seguinte commando:

\begin{lstlisting}
curl -fLo ~/.vim/autoload/plug.vim --create-dirs \textbackslash
    https://raw.githubusercontent.com/junegunn/vim-plug/master/plug.vim
\end{lstlisting}

Para adicionar plugins, adicionam-se o repositório (normalmente do GitHub) do plugin ao vimrc com a seguinte linha:

\begin{lstlisting}
Plug 'repo-owner/repo-name'
\end{lstlisting}

Depois disto, para instalar os plugins, é necessário correr o comando \texttt{:PlugInstall} dentro do Vim.

Para atualizar e instalar os plugins em falta, usa-se \texttt{:PlugUpdate}.

Existem ainda outros comandos que podem ser vistos na página do GitHub do Vim-Plug.

Para configurar os plugins, basta adicionar as configurações ao vimrc.\ As configurações podem ser vistas na página do GitHub do plugin em questão.

Para uma configuração de exemplo, veja o vimrc no repositório \href{https://github.com/GLUA-UA/vim-workshop/blob/master/.vimrc}{GLUA-UA/vim-workshop}.
